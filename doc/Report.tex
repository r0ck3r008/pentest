%  Make this into a pdf document as follows:
%
%
% The edit the Report.tex file appropriately to include only those elements that
% make sense for the assignment you're reporting on.
%
% You can use a tool like TeXShop or Texmaker or some other graphical tool
% to convert Report.text into a pdf file.
%
% If you are making this with command line tools, you'd run the following command:
%
%     latex Report.tex
%
% That will generate a dvi (device independent) document file called Report.dvi
% The pages reported in the table of contents won't be correct, since latex only
% processes one pass over the document. To adjust the page numbers in the contents,
% run latex again:
%
%    latex Report.tex
%
% Then run
%
%   dvipdf Report.dvi
%
% to generate Report.pdf
%
% You can view this file to check it out by running
%
% xdg-open Report.pdf
%
% That's it.
  

\documentclass[notitlepage]{article}

\usepackage{bibunits}
\usepackage{comment}
\usepackage{graphicx}
\usepackage{amsmath}
\usepackage{datetime}
\usepackage{numprint}

% processes above options
\usepackage{palatino}  %OR newcent ncntrsbk helvet times palatino
\usepackage{url}
\usepackage{footmisc}
\usepackage{endnotes}

\setcounter{secnumdepth}{0}
\begin{document}

\nplpadding{2}
\newdateformat{isodate}{
  \THEYEAR -\numprint{\THEMONTH}-\numprint{\THEDAY}}
  
\title{Penetration Test Report Title}
\author{Your Name}
\date{\isodate\today}

\maketitle

\tableofcontents

\newpage
\section{Executive Summary}

Here's how to include a picture:

\includegraphics[width=4in]{foo.png}

This is the executive summary.
I like it.


%  \subsection{Background}

%  \subsection{Overall Posture}

%  \subsection{Risk Ranking/Profile}

%  \subsection{General Findings}

%  \subsection{Recommendation Summary}

%  \subsection{Strategic Roadmap}

\section{Technical Report}

%  \subsection{Introduction}

% Include one of these headings for each finding.

  \subsection{Finding: Description of finding}

	\subsubsection{Risk Rating}
		Here you identify the risk	 and point out the potential outcome of
		exploitation of this vulnerability.
		
  	\subsubsection{Vulnerability Description}
  		Here you provide a brief description of the nature of the vulnerability.
  		
  	\subsubsection{Confirmation method}
		This section contains the information necessary to verify that the
		vulnerability still exists. Note that inability to confirm the vulnerability
		does not exists using this method does not guarantee that the
		vulnerability has been addressed or mitigated.
		
    \subsubsection{Mitigation or Resolution Strategy}
    	This is where you describe how to address the problem.
    	Can it be completely solved or can you, at least, reduce the likelihood
    	that the vulnerability can be exploited?
		
\section{Attack Narrative}

	In this section you provide enough information for a knowledgeable
	penetration tester to reproduce your results. Thus, if active intelligence
	was gathered, you provide enough details of the methods employed so
	that a knowledgeable penetration tester could gather the same information.
	If a vulnerability
	was exploited, you provide enough details of your activities so that a
	knowledgeable penetration tester could reproduce your results under the
	same circumstances.<

\end{document} 
