\documentclass[10pt,a4paper]{article}

\usepackage[a-1b]{pdfx} %high quality pdf
\usepackage{datetime}
\usepackage{numprint}
\usepackage{palatino}
\usepackage{authblk}
\usepackage[margin=0.75in]{geometry}
\usepackage{hyperref}
\usepackage{graphicx}
\usepackage{multicol}

\pdfgentounicode=1

\setlength{\parindent}{2em}
%\setlength{\parskip}{1em}
\renewcommand{\baselinestretch}{1.25}

\begin{document}

\nplpadding{2}
\newdateformat{isodate}{\THEYEAR-\numprint{\THEMONTH}-\numprint{\THEDAY}}

\title{Penetration Testing Report for Ex070\\ \large{\textit{Codename: `BriansService`}}}
\author{Naman Arora\\\small{Pr0b3 LLC}}
\date{\isodate\today}

\maketitle
\section{Executive Summary}
\textit{Pr0b3 LLC.} has been tasked to conduct a penetration test of the network and other technical infrastructure of \textit{F4rmc0rp}.
During one such assessment, a vulnerable service was discovered in the \textit{www.f4rmc0rp.com} domain at a non-well-known port \textbf{1337}.
This service identifies as \textit{"waste"} during an \textit{nmap scan} of the first 2000 ports of the host.
The service compromises the integrity and security of the host and immediate remediation of the found vulnerabilities is highly recommended.

\section{Technical Report}
This service is vulnerable to three major vulnerabilities.
\subsection{Findings}
\begin{figure}[h!]
	\includegraphics[width=\textwidth]{pics/discovery.png}
	\caption{\textit{Nmap Scan} for Vulnerable Service at Port 1337}
\end{figure}
\begin{itemize}
	\item{Buffer Overflow (Score: \textbf{10.0})}\\
		This service is vulnerable to \textit{Buffer Overflow} attack which may lead an attacker to remote shell access (Figure \ref{shell}).
	\item{Cleartext Login and Information Exchange (Score: \textbf{4.5})}\\
		The service has no built in encryption and all the packets can be captured to harvest the username as well as information being transferred to a authorized user (Figure \ref{cleartext}).
	\item{Missing Password Authentication (Score: \textbf{4.0})}\\
		This service is not password protected and any attacker with a bit of social engineering can figure out that the username is \textit{brian}.
		Once the login is successful, attacker gets access to three commands to gain information about the system (Figure \ref{passwd}).
\end{itemize}
\begin{figure}[h!]
	\includegraphics[width=\textwidth]{pics/shell.png}
	\caption{Remote Shell Access \textit{Buffer Overflow}}
	\label{shell}
\end{figure}
\begin{figure}[h!]
	\includegraphics[width=\textwidth]{pics/cleartext.png}
	\caption{Username in Cleartext \textit{(Packet Capture)}}
	\label{cleartext}
\end{figure}
\begin{figure}[h!]
	\includegraphics[width=\textwidth]{pics/passwd.png}
	\caption{No Password Authentication}
	\label{passwd}
\end{figure}

\newpage
\subsection{Attack Narrative}
The attacker may gain insight to the username by following posts on social media posted by Brian Oppenheimer using official \textit{f4rmc0rp.com} email address referring to this service or by sniffing and capturing packets while the service is being used by authorized user, since the information is in cleartext.
Once the attacker has access to the username, since no password is required for login, attacker can gain access to commands designed to be used by system administrators.
The connection to the service can be established by using simple \textit{telnet} or \textit{netcat} utilities.

This access to the username can also lead an attacker to exploit an underlying buffer overflow vulnerability.
The vulnerability can be exploited by passing 32 random characters in the username field followed by any command of choice to run on the target host.
Once entered, the service prompts again for username, which when entered as \textit{brian}, gives access to the previously entered command.

\subsection{Remediation}
The service can be made secure against the mentioned vulnerabilities by,
\begin{itemize}
	\item{Adding Secure Password Authentication}\\
		This can be done by either using third part libraries or coding the password field feature during authentication and maintaining a database of securely hashed passwords.
	\item{Using Encryption}\\
		This can be done by using APIs like \textit{libressl} or \textit{openssl} in the source code.
		Make sure secure functionalities like Encryption should \textbf{not} developed from scratch and secure and well established libraries should be used instead.
	\item{Using correct buffer length when taking username input}\\
		The access to source code for the service was found during the exfiltration phase.
		An analysis of the code reveals a mistyped buffer length in the \textit{fgets call} on \textit{admin} buffer.
		The correct length is represented by the variable \textit{NAMELEN = 16 bytes} while \textit{BUFLEN = 1024 bytes} is used instead.
\end{itemize}

\end{document}
