\documentclass[10pt,a4paper]{article}

\usepackage[a-1b]{pdfx} %high quality pdf
\usepackage{datetime}
\usepackage{numprint}
\usepackage{palatino}
\usepackage{authblk}
\usepackage[margin=0.75in]{geometry}
\usepackage{listings}
\usepackage{hyperref}
\usepackage{graphicx}
\usepackage{multicol}

\pdfgentounicode=1

%paragraph indentation
\setlength{\parindent}{2em}
%\setlength{\parskip}{1em}
\renewcommand{\baselinestretch}{1.25}

\begin{document}

\nplpadding{2}
\newdateformat{isodate}{\THEYEAR-\numprint{\THEMONTH}-\numprint{\THEDAY}}

\title{Penetration Testing Report for Ex030\\ \large{\textit{Codename: `DNS Reconnaissance`}}}
\author{Naman Arora}
\date{\isodate\today}

\maketitle
\section{Technical Report}
\subsection{Findings}
During the DNS Reconnaissance phase, the following IP blocks were uncovered:
\begin{itemize}
	\item{10.30.0.0/24}
	\item{172.30.0.0/24}
\end{itemize}
Within these IP blocks, the following 11 hosts were spotted to be alive:
\begin{multicols}{2}
\begin{itemize}
	\item{www.f4rmc0rp.com}\\ 172.30.0.128
	\item{innerouter.f4rmc0rp.com}\\ 172.30.0.3
	\item{ns.f4rmc0rp.com}\\ 172.30.0.128
	\item{mail.f4rmc0rp.com}\\ 172.30.0.130
	\item{pop.f4rmc0rp.com}\\ 172.30.0.128
	\item{devbox.f4rmc0rp.com}\\ 10.30.0.32
	\item{pdc.f4rmc0rp.com}\\ 10.30.0.90
	\item{patronum.f4rmc0rp.com}\\ 10.30.0.97
	\item{herd.f4rmc0rp.com}\\ 10.30.0.98
	\item{KEY005-IHIHIWJRhzMTH4qXCCwOuA.f4rmc0rp.com}\\ 10.30.0.102
	\item{linuxserver.f4rmc0rp.com}\\ 10.30.0.128
\end{itemize}
\end{multicols}

\subsection{Attack Narrative}

\subsubsection{Task 01}
The first task is to scan the \textit{www.f4rmc0rp.com} website with \textit{fierce} domain scanner and note down all the domains that it finds.
The following command was issued:\\
\begin{lstlisting}[language=bash]
	$> fierce -dns f4rmc0rp.com
\end{lstlisting}
\begin{figure}[!h]
	\includegraphics[width=\textwidth,scale=0.15]{pentest/fierce_simple.png}
	\caption{Fierce Simple Run}
\end{figure}

\newpage
\subsubsection{Task 02}
The second task was to inspect the source code of the \textit{fierce} scanner to look for its default wordlist.
First, to locate the \textit{fierce} executable,
\begin{lstlisting}[language=bash]
	$> which fierce
\end{lstlisting}
which locates it in \textit{/usr/bin/fierce}.
Then on inspecting the file type of \textit{/usr/bin/fierce} usig \textit{file},
\begin{lstlisting}[language=bash]
	$> file /usr/share/fierce
\end{lstlisting}
it reveals that its a PERL script. Opening it with \textit{vim} and searching for wordlist reveals a file \textit{/usr/share/fierce/hosts.txt}.
\begin{figure}[!h]
	\includegraphics[width=\textwidth]{pentest/fierce_wordlist_hosts.png}
	\caption{Fierce Wordlist}
\end{figure}

\newpage
\subsubsection{Task 03}
The next task was to run \textit{CeWL} against \textit{www.f4rmc0rp.com} to generate a custom word list and use it with unison with the fierce wordlist to find more subdomains.
To execute CeWL,
\begin{lstlisting}[language=bash]
	$> cewl -d 3 -o -w cewl_list.txt www.f4rmc0rp.com
\end{lstlisting}
\begin{figure}[!h]
	\includegraphics[width=\textwidth]{pentest/cewl_wordlist.png}
	\caption{CeWL Wordlist}
\end{figure}

Now using this wordlist with fierce,
\begin{lstlisting}
	$> fierce -wordlist cewl_list.txt -dns f4rmc0rp.com
\end{lstlisting}
\begin{figure}[!h]
	\includegraphics[width=\textwidth]{pentest/fierce_cewl_run.png}
	\caption{Fierce Run with CeWL Wordlist}
\end{figure}


On combining a reverse lookup method provided by \textit{fierce} and using the default \textit{fierce} wordlist merged with the one produced by \textit{CeWL},
\begin{lstlisting}[language=bash]
	$> cat /usr/share/fierce/hosts.txt ./cewl_list.txt > cewl_fierce_list.txt
	$> fierce -wordlist cewl_fierce_list.txt -dns f4rmc0rp.com -traverse 4 -threads 4
\end{lstlisting}
\newpage
This produces a total of 11 hosts, 3 more than combined total of \textit{fierce} default run and \textit{fierce} with \textit{CeWL} wordlist run.
\begin{figure}[h!]
	\includegraphics[width=\textwidth]{pentest/fierce_cewl_traverse_run.png}
	\caption{Fierce Final Run}
\end{figure}

\newpage
\subsubsection{Task 04}
The next task was to determine how \textit{fierce} could have enumerated the hosts that it did.
The three methods used were \textit{Reverse Lookup}, \textit{from CeWL wordlist}, \textit{default fierce wordlist}.
The following observations were made,
\begin{multicols}{2}
\begin{itemize}
	\item{www.f4rmc0rp.com 172.30.0.128}\\
		Via default \textit{fierce} wordlist.
	\item{innerouter.f4rmc0rp.com 172.30.0.3}\\
		Via reverse lookup.
	\item{ns.f4rmc0rp.com 172.30.0.128}\\
		Via default \textit{fierce} wordlist.
	\item{mail.f4rmc0rp.com 172.30.0.130}\\
		Via default \textit{fierce} wordlist.
	\item{pop.f4rmc0rp.com 172.30.0.128}\\
		Via default \textit{fierce} wordlist.
	\item{devbox.f4rmc0rp.com 10.30.0.32}\\
		Via reverse lookup.
	\item{pdc.f4rmc0rp.com 10.30.0.90}\\
		Via default \textit{fierce} wordlist.
	\item{patronum.f4rmc0rp.com 10.30.0.97}\\
		Via \textit{CeWL} wordlist.
	\item{herd.f4rmc0rp.com 10.30.0.98}\\
		Via reverse lookup.
	\item{KEY005-IHIHIWJRhzMTH4qXCCwOuA.f4rmc0rp.com 10.30.0.102}\\
		Via reverse lookup.
	\item{linuxserver.f4rmc0rp.com 10.30.0.128}\\
		Via reverse lookup.
\end{itemize}
\end{multicols}
The conclusions that reverse lookups were used by \textit{fierce} was derived by analyzing the traffic using the \textit{tcpdump}.

\newpage
\subsubsection{Task 05}
The very last task is to comapre the results obtained from a combination of all three types of lookups to the results from amass scanner.
To execute amass scanner,
\begin{lstlisting}[language=bash]
	$> amass enum -r ns.f4rmc0rp.com -d f4rmc0rp.com -ip -brute -src
\end{lstlisting}
was used.
Amass was limited since it derives it \textit{`wordlist`} from well known DNS servers present on the internet which its restricted from accessing.
It could only find 5 domains.
\begin{figure}[!h]
	\includegraphics[width=\textwidth]{pentest/amass_run1.png}
	\caption{Amass Run}
\end{figure}
The team meeting URL \href{https://meet.jit.si/ex0x30}{https://meet.jit.si/ex0x30}.

\subsection{Risk Analysis}
The risk involved so far in the information obtained is minimal at maximum.
DNS zone transfer was attempted which failed promptly during each type of analysis.

\end{document}
