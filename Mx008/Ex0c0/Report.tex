\documentclass[10pt,a4paper]{article}

\usepackage[a-1b]{pdfx} %high quality pdf
\usepackage{datetime}
\usepackage{numprint}
\usepackage{palatino}
\usepackage{authblk}
\usepackage[margin=0.75in]{geometry}
\usepackage{hyperref}
\usepackage{graphicx}
\usepackage{titlesec}
\usepackage{listings}

\pdfgentounicode=1

\setlength{\parindent}{2em}
%\setlength{\parskip}{1em}
\renewcommand{\baselinestretch}{1.1}

\begin{document}

\nplpadding{2}
\newdateformat{isodate}{\THEYEAR-\numprint{\THEMONTH}-\numprint{\THEDAY}}

\title{Penetration Testing Report for Ex0c0\\ \large{\textit{Codename: `LiftingTheVeil`}}}
\author{Naman Arora\\\small{Pr0b3 LLC}}
\date{\isodate\today}

\maketitle
\section{Attack Narrative}
The following pieces of information can be used to build up on the progress done in the vulnerability assessment of the \textit{F4rmc0rp} cyber infrastructure,
\begin{itemize}
	\item{Administrative access to \textit{PFSense} control panel.}
	\item{Access to HRED\textbackslash Brian account.}
	\item{Access to \textit{n.nomen} user credentials.}
\end{itemize}

A PowerShell script/\textit{GoLang} binary can be generated using the \textit{Veil} framework for \textit{reverse\_https} exploitation which is resilient to antimalware detection (Fig. \ref{veil_evade}).
The binary can then be transported via a network drive mapped from \textit{`kali`@`172.24.0.10`} to \textit{`herd`@`10.30.0.98`} and further from \textit{`HERD`@`10.30.0.98`} to \textit{`PATRONUM`@`10.30.0.97`}.
Login credentials of \textit{`n.nomen`} are viable for \textit{`PATRONUM`@`10.30.0.97`}.
\begin{figure}[h!]
	\includegraphics[width=\textwidth]{pics/veil_evade.png}
	\caption{\textit{Veil} Evading AntiMalware}
	\label{veil_evade}
\end{figure}

Once the exploitation script/binary is successfully transported, a \textit{reverse\_https} handler can be started on the \textit{Metasploit} on \textit{`kali`@`172.24.0.10`} (Fig. \ref{patronum_conn}). On the execution of script/binary on \textit{`PATRONUM`@`10.30.0.97`}, a session sets up giving access to \textit{`PATRONUM`@`10.30.0.97`} as the user \textit{n.nomen}.
On gaining access to \textit{`PATRONUM`@`10.30.0.97`}, \textit{`netstat`} command on the meterpreter terminal provides the open/listening port list (Fig \ref{patronum_ports} and \ref{herd_ports}).
\begin{figure}[h!]
	\includegraphics[width=\textwidth]{pics/patronum_ports.png}
	\caption{Reverse HTTPS \textit{Meterpreter} from \textit{`PATRONUM`@`10.30.0.97`}}
	\label{patronum_conn}
\end{figure}
\begin{figure}[h!]
	\includegraphics[width=\textwidth]{pics/patronum_ports.png}
	\caption{Open ports on \textit{`PATRONUM`@`10.30.0.97`}}
	\label{patronum_ports}
\end{figure}
\begin{figure}[h!]
	\includegraphics[width=\textwidth]{pics/herd_ports.png}
	\caption{Open ports on \textit{`HERD`@`10.30.0.98`}}
	\label{herd_ports}
\end{figure}

On further exploration, if a RDP port is forwarded from \textit{`PATRONUM`@`10.30.0.97`} to the \textit{`innerrouter`@`172.30.0.3`} using the \textit{PFSense} control panel, a remote desktop connection from \textit{`kali`@`172.24.0.10`} fails, unlike when similar process is done for \textit{`HERD`@`10.30.0.98`}.
One possible explanation for such an observation could be that \textit{`PATRONUM`@`10.30.0.97`} restricts the possible RDP sessions by IP addresses within the company network and/or another restriction might be in place which disallows everyone except some user groups within the organization.

The team meeting link: \href{https://ufl.zoom.us/meeting/register/tJArdOCsqj0pHtZV1nsOgBgYZjtfDycf2rhG}{https://ufl.zoom.us/meeting/register/tJArdOCsqj0pHtZV1nsOgBgYZjtfDycf2rhG}.

\end{document}
