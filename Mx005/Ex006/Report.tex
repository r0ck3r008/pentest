\documentclass[10pt,a4paper]{article}

\usepackage[a-1b]{pdfx} %high quality pdf
\usepackage{datetime}
\usepackage{numprint}
\usepackage{palatino}
\usepackage{authblk}
\usepackage[margin=0.75in]{geometry}
\usepackage{hyperref}
\usepackage{graphicx}
\usepackage{multicol}

\pdfgentounicode=1

\setlength{\parindent}{2em}
%\setlength{\parskip}{1em}
\renewcommand{\baselinestretch}{1.25}

\begin{document}

\nplpadding{2}
\newdateformat{isodate}{\THEYEAR-\numprint{\THEMONTH}-\numprint{\THEDAY}}

\title{Penetration Testing Report for Ex060\\ \large{\textit{Codename: `OpenVAS`}}}
\author{Naman Arora\\\small{Pr0b3 LLC}}
\date{\isodate\today}

\maketitle
\section{Executive Summary}
\subsection{Introduction}
\textit{`Pr0b3 LLC`} has been hired by \textit{`F4rmc0rp`} for penetration testing and vulnerability analysis of the technical infrastructure.
The analysis is conducted in a manner that simulates a real life intruder both in terms of knowledge and access to the network and related resources.

\subsection{Result Summary}
The initial analysis exposes a glaring vulnerability in the system that warrants \textbf{immediate and expedited} response as well as some other lower risk vulnerabilities.
On examining the services that the host \textit{`www.f4rmc0rp.com`} provides, it revealed that the \textit{FTP} service hosted is vulnerable to a malicious actor gaining remote access to the host.
This remote access, if exploited, may lead to further exploitation of the network and/or sensitive data exfiltration.
Data exfiltration has been proven possible in subsequent exploitation simulation against the host.

\section{Technical Report}
The \textit{www.f4rmc0rp.com} server was analyzed for securities vulnerabilities and the following findings were gathered,
\subsection{vsftpd v2.3.4 Vulnerable to Backdoor Entry}
\begin{itemize}
        \item{Risk Rating}\\
                10.0
        \item{Vulnerability Description}\\
                The FTP server program is vulnerable to remote backdoor entry into the server.
		The vulnerability enables malicious actors to exfiltrate sensitive data from the host.
		To simulate this behavior, \textit{`KEY008`}, a nonsensitive file was exfiltrated which contains the string,
		\textbf{\textit{`KEY008:0KAPlIo8G6+4wLP9prlZRg==`}}.
        \item{Confirmation Method}\\
                \textit{OpenVAS} scan confirms the existence of this vulnerability.
                Exploit for the access is available in the \href{https://github.com/rapid7/metasploit-framework/blob/master/modules/exploits/unix/ftp/vsftpd_234_backdoor.rb}{Metasploit Framework}.
        \item{Mitigation Strategy}\\
                Upgrade to the latest version of the software.
\end{itemize}
\subsection{vsftpd v2.3.4 FTP Unencrypted Cleartext Login}
\begin{itemize}
        \item{Risk Rating}\\
                4.8
        \item{Vulnerability Description}\\
                FTP Login credentials are sent in cleartext when loging in.
                A packer sniffer may harvest legitimate login credentials.
        \item{Confirmation Method}\\
                \textit{OpenVAS} scan confirms the vulnerability.
                Packet capture using \textit{tcpdump} or \textit{wireshark} also can confirm this.
        \item{Mitigation Strategy}\\
                Enable secure FTP, \textit{FTPS} or enforce connection via \textit{`Auth TLS`} command.
\end{itemize}
\subsection{Anonymous FTP Login}
\begin{itemize}
        \item{Risk Rating}\\
                6.4
        \item{Vulnerability Description}\\
                Anonymous login is possible to the FTP server and exfiltration of \textit{`hereisafile`} is possible.
        \item{Confirmation Method}\\
                \textit{OpenVAS} scan or \textit{nmap} aggressive scan confirms this vulnerability.
        \item{Mitigation Strategy}\\
                Disable Anonymous login in configuration files.
\end{itemize}
\subsubsection{TCP Timestamps}
\begin{itemize}
        \item{Risk Rating}\\
                2.6
        \item{Vulnerability Description}\\
                It is possible to guess the uptime of the host.
        \item{Confirmation Method}\\
                \textit{OpenVAS} scan or \textit{nmap} OS scan confirms this vulnerability.
        \item{Mitigation Strategy}\\
                Disable TCP Timestamps.
\end{itemize}

\newpage
\subsection{Attack Narrative}
The version 2.3.4 of vsftpd has a malicious backdoor code segment in its download archive.
The vulnerability is triggered by the presence of two values \textit{viz.} \textit{0x29 and 0x3A} which in ASCII is \textit{`:)`} in the \textit{USER} field while logging in (\textit{refer} \href{https://pastebin.com/AetT9sS5}{\textit{pastebin.com}} for \textit{code diff}).
This trigger opens up a TCP port \textit{6200} on the target host for backdoor connection.
The \textit{stdin, stdout and stderr} are closed and duplicated as the \textit{fd} of the accepted socket connection hence giving full shell access (Fig. \ref{ftp_redo}).
The packet capture of FTP USER field provides an insight into how this is possible, refer Fig. \ref{pkt_vuln}.
Again, such a packet capture is possible because this instance of vsftpd has \textit{`Unencrypted Cleartext Login`} as one of its other vulnerabilities.
The team meeting URL is \href{https://ufl.zoom.us/j/93976394163?pwd=dHllQ0xUVnhMMDIyOWNFa1Vzdkk3QT09}{this}.
\begin{figure}[h!]
        \includegraphics[width=\textwidth]{pics/ftp_redo.png}
	\caption{Trigger Using Shell (\textit{www.f4rmc0rp.com} resolves to \textit{172.30.0.128})}
	\label{ftp_redo}
\end{figure}
\begin{figure}[h!]
        \includegraphics[width=\textwidth]{pics/pkt_vuln.png}
        \caption{Malicious Packet Capture}
	\label{pkt_vuln}
\end{figure}

\end{document}
