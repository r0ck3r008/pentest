\documentclass[10pt,a4paper]{article}

\usepackage[a-1b]{pdfx} %high quality pdf
\usepackage{datetime}
\usepackage{numprint}
\usepackage{palatino}
\usepackage{authblk}
\usepackage[margin=0.75in]{geometry}
\usepackage{hyperref}
\usepackage{graphicx}
\usepackage{titlesec}
\usepackage{listings}

\pdfgentounicode=1

\setlength{\parindent}{2em}
%\setlength{\parskip}{1em}
\renewcommand{\baselinestretch}{1.1}

\begin{document}

\nplpadding{2}

\title{Penetration Testing Report for Ex120\\ {\large\textit{Codename: `BriansProject'}}}
\author{Naman Arora}
\date{\today}

\maketitle
\section*{Technical Report}
The following vulnerabilities were discovered by visiting the web site hosted at \textit{`http://www.f4rmc0rp.com/brian`} on the host \textit{`www'}@172.30.0.128.
\subsection*{Finding: Potential \textit{`htpasswd'} Leak}
        \subsubsection*{Risk Rating: 7.5}
        The \textit{`htpasswd'} file is available for access by any unauthenticated user.
        \subsubsection*{Vulnerability Description}
        The protected \textit{`htpasswd'} file is stored within the website hosted at \textit{`www.f4rmc0rp.com/brian'}.
        This might enable unauthenticated visitors to gain access to user \textit{`brian's'} hashed credentials.
        The hashed credentials can then easily be cracked using brute force cracking to gain clear text passwords (Fig \ref{john}).
        The same credentials have been verified to provide \textit{`SSH'} access to the server hosting the website under the user \textit{`brian'}.
        \subsubsection*{Confirmation Method}
        The file can be accessed by visiting the URL \textit{`http://www.f4rmc0rp.com/brian/i*******/htpasswd`} (Fig \ref{htpasswd}).
        \subsubsection*{Mitigation Strategy}
        \textit{`htpasswd'} file should never be present in the directory structure of the enabled website (\href{https://httpd.apache.org/docs/2.4/en/programs/htpasswd.html}{Apache Advisory}).
        It should immediately be placed under the host's regular non-accessible part of file system like \textit{`/usr/local/etc/apache/.htpasswd`}.
\begin{figure}[h!]
	\includegraphics[width=\textwidth]{pics/htpasswd.png}
        \caption{\textit{`htpasswd'} Access}
        \label{htpasswd}
\end{figure}
\subsection*{Finding: Unprotected Access to Administrative Page}
        \subsubsection*{Risk Rating: 5.0}
        The access to the upload page, which is supposed to be password protected, is accessible without authentication.
        \subsubsection*{Vulnerability Description}
        Such an access provides a way for any unauthenticated visitor to upload images to the webserver.
        This can, in tandem with other vulnerabilities, be used to eventially take over the control of the website by uploading a malicious \textit{php} script.
        \subsubsection*{Confirmation Method}
        The unauthenticated access can be gained by visiting \textit{`http:/f4rmc0rp.com/brian/imgfiles/upload.php}' (Fig \ref{upload}).
        \subsubsection*{Mitigation Strategy}
        A \textit{`htaccess'} file must be added to the \textit{`imgfiles'} directory protecting the access to \textit{`upload.php'}.
\begin{figure}[h!]
	\includegraphics[width=\textwidth]{pics/upload.png}
        \caption{Unauthenticated Access to Administrative Panel}
        \label{upload}
\end{figure}
\subsection*{Finding: Client Side File Upload Validation}
        \subsubsection*{Risk Rating: 10}
        A user with access to upload page may upload any file by bypassing client side validation.
        \subsubsection*{Vulnerability Description}
        A user with access to upload page may use proxy to upload any type of file, including php scripts, to the server.
        These file can then be visited from the browser to provide access to the host as the user \textit{`www-data'}.
        \subsubsection*{Confirmation Method}
        The page source of \textit{`imgfiles/upload.php'} has javascript validation built in (Fig \ref{js_valid}).
        \subsubsection*{Mitigation Strategy}
        The validation should always be done on the side of the server, \textit{i.e.} using PHP and not the javascript in this case.
\begin{figure}[h!]
        \includegraphics[width=\textwidth]{pics/js_valid.png}
        \caption{Client side upload validation}
        \label{js_valid}
\end{figure}

\section*{Attack Narrative}
\begin{figure}[h!]
        \includegraphics[width=\textwidth]{pics/brian_john.png}
        \caption{Brute Forcing user \textit{`brian'} password using john}
        \label{john}
\end{figure}
An unauthenticated user may visit the URL \textit{`http://www.f4rmc0rp.com/brian/i*******/htpasswd`} and gain access to the sensitive \textit{`htpasswd`} file (Fig \ref{htpasswd}).
Once the hash of the user \textit{brian} is accessed, it can be brute forced within seconds using \textit{John} to get the plain text password (Fig \ref{john}) using the command.
\begin{lstlisting}[language=bash]
	>> john htpasswd
\end{lstlisting}

By having unauthenticated access to the website \textit{`http://f4rmc0rp.com/brian`}, an unauthenticated user can access the administrative panel which is not password protected (Fig \ref{upload}).
On auditing the source for the admin-panel, \textit{`www.f4rmc0rp.com/imgfiles/upload.php'}, a Burp Suit proxy can be set up to intercept the traffic to the website (Fig \ref{js_valid}).
While uploading the file, a php script can be masqueraded as a png/jpg file to pass the javascript based client side validation and can be restored with original extension when intercepted by the Burp Suit proxy (Fig \ref{php} and \ref{bsuit}).
\begin{figure}[h!]
        \includegraphics[width=\textwidth]{pics/php_shell.png}
        \caption{Php Script to run shell commands using GET}
        \label{php}
\end{figure}
\begin{figure}[h!]
        \includegraphics[width=\textwidth]{pics/bsuit.png}
        \caption{Burp Suit proxy intercept and change extension back to \textit{`php'} after passing client side validation}
        \label{bsuit}
\end{figure}

Once the file is uploaded, the access to the \textit{`private`} directory can be set to permissive which is located one level up from the current directory (Fig \ref{chmod}).
\begin{figure}[h!]
        \includegraphics[width=\textwidth]{pics/chmod_cmd.png}
        \caption{Chmod command to provide permissive access}
        \label{chmod}
\end{figure}
After setting the permissive access to the \textit{private} directory, the credentials gained by brute forcing of the user \textit{brian} can be used to gain \textit{SSH} access on the host \textit{`www'}@172.30.0.128 under the same user.
Once done, the directory \textit{`/var/www/html/brian/private`} can be visited and access to private file \textit{`key20'} can be gained (Fig \ref{key20}).
\begin{figure}[h!]
        \includegraphics[width=\textwidth]{pics/key020.png}
        \caption{Access to private file \textit{`key20`}}
        \label{key20}
\end{figure}
\end{document}
