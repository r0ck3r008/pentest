\documentclass[10pt,a4paper]{article}

\usepackage[a-1b]{pdfx} %high quality pdf
\usepackage{datetime}
\usepackage{numprint}
\usepackage{palatino}
\usepackage{authblk}
\usepackage[margin=0.75in]{geometry}
\usepackage{hyperref}
\usepackage{graphicx}
\usepackage{titlesec}
\usepackage{listings}

\pdfgentounicode=1

\setlength{\parindent}{2em}
%\setlength{\parskip}{1em}
\renewcommand{\baselinestretch}{1.1}

\begin{document}

\nplpadding{2}
\newdateformat{isodate}{\THEYEAR-\numprint{\THEMONTH}-\numprint{\THEDAY}}

\title{Penetration Testing Report for Ex100\\ \large{\textit{Codename: `Responder'}}}
\author{Naman Arora}
\date{\isodate\today}

\maketitle
\section*{Finding: Potential Credential Harvesting by Malicious WPAD Responses}
\subsection*{Risk Rating: 6.0}
A malicious actor may respond with malicious proxy settings to the WPAD configuration requests.
Also, due to availability of Basic authentication, credentials can then be harvested.
\subsection*{Vulnerability Description}
The host \textit{`PATRONUM'@10.30.0.97} uses WPAD protocol to request for browser proxy configuration.
Also, the host allows Basic authentication to be forced for login.
A malicious actor may respond with proxy configuration requests and configure a proxy that it controls.
In such an event, if Basic Authentication is forced, credentials of the user can harvested.
\subsection*{Confirmation Method}

\subsection*{Remediation}
Disable Basic Authentication as an accepted option. In addition to that, disable WPAD configuration protocol is possible.

\section*{Attack Narrative}
With root access to \textit{`devbox`@10.30.0.32} using the user credentials of the user \textit{`m.mason`}, \textit{Responder} tool can be uploaded to the host.
Once \textit{Responder} is available on the host, \textit{apache2} and \textit{bind9} services can be disabled using,
\begin{lstlisting}[language=bash]
	>> systemctl stop apache2
	>> systemctl stop bind9
\end{lstlisting}
Having done that, \textit{Responder} can be run on the interface \textit{ens33} using the command,
\begin{lstlisting}[language=bash]
	>> ./Responder -I ens33 -bw
\end{lstlisting}
After sometime, login credentials of the user \textit{`m.mason'} start to get captured (Fig \ref{creds}).
\begin{figure}[h!]
	\includegraphics[width=\textwidth]{pics/creds.png}
        \caption{Captured Credentials}
        \label{creds}
\end{figure}

Once the credentials have been captured, a quick scan of hosts on the \textit{`10.30.0.0/24`} network reveals that \textit{`pdc`@10.30.0.90} has SMB shares enabled (Fig \ref{pdc_nmap}).
On forwarding port 445 to \textit{`172.30.0.3`} from \textit{`pdc`@10.30.0.90}, \textit{`kali'@172.24.0.10} can login to SMB Share and ex filtrate sensitive information.
The following command can be used to list all shares on \textit{`pdc`@10.30.0.90} (Fig \ref{shares}).
\begin{lstlisting}[language=bash]
	>> smbclient -L 172.30.0.3 -U m.mason
\end{lstlisting}
\begin{figure}[h!]
	\includegraphics[width=\textwidth]{pics/pdc_nmap.png}
        \caption{Nmap scan of \textit{`pdc`@10.30.0.90} from \textit{`devbox`@10.30.0.32}}
        \label{pdc_nmap}
\end{figure}
\begin{figure}[h!]
	\includegraphics[width=\textwidth]{pics/smb_share_list.png}
        \caption{List of SMB Shares}
        \label{shares}
\end{figure}


The following command can be used to access the shares, \textit{`MasonShare`},  \textit{`SYSVOL`}, \textit{`NETLOGON`} and \textit{`IPC\$'} (Fig \ref{netlogon}, \ref{mason}, \ref{sysvol} and \ref{ipc}).
\begin{lstlisting}[language=bash]
	>> smbclient \\\\172.30.0.3\\MasonShare -U m.mason
	>> smbclient \\\\172.30.0.3\\SYSVOL -U m.mason
	>> smbclient \\\\172.30.0.3\\NETLOGON -U m.mason
	>> smbclient \\\\172.30.0.3\\IPC$ -U m.mason
\end{lstlisting}
\begin{figure}[h!]
	\includegraphics[width=\textwidth]{pics/mason.png}
        \caption{MasonShare Share Access}
        \label{mason}
\end{figure}
\begin{figure}[h!]
	\includegraphics[width=\textwidth]{pics/netlogon.png}
        \caption{NETLOGON Share Access}
        \label{netlogon}
\end{figure}
\begin{figure}[h!]
	\includegraphics[width=\textwidth]{pics/sysvol.png}
        \caption{SYSVOL Share Access}
        \label{sysvol}
\end{figure}
\begin{figure}[h!]
	\includegraphics[width=\textwidth]{pics/ipc.png}
        \caption{IPC\$ Share Access}
        \label{ipc}
\end{figure}

\end{document}
