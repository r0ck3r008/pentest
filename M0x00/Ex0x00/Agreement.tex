\documentclass[10pt,a4paper]{article}

\usepackage[a-1b]{pdfx} %high quality pdf
\usepackage{datetime}
\usepackage{numprint}
\usepackage{palatino}
\usepackage{authblk}
\usepackage[margin=1in]{geometry}
\usepackage[most]{tcolorbox} % Rational colorbox

\pdfgentounicode=1

\definecolor{aliceblue}{rgb}{0.98, 0.98, 1} % colorbox color

\newcommand\signature[2]{% Name; Department
\noindent\begin{minipage}{5cm}
    \noindent\vspace{3cm}\par
    \noindent\rule{5cm}{1pt}\par
    \noindent\textbf{#1}\par
    \noindent#2%
\end{minipage}}

%paragraph indentation
\setlength{\parindent}{4em}
%\setlength{\parskip}{1em}
\renewcommand{\baselinestretch}{1.0}

\begin{document}

\nplpadding{2}
\newdateformat{isodate}{\THEYEAR-\numprint{\THEMONTH}-\numprint{\THEDAY}}

\title{Scoping and Rules of Engagement}
\author{Naman Arora}
\affil{\small{Pr0b3 LLC., \\Gainesville, Fl}}
\date{\isodate\today}

\maketitle

\tableofcontents

\newpage
\section{Prelude}
This document is intended to serve as an agreement for a Penetration Test/Vulnerability Assessment of \textit{'F4rmc0rp`}, hereinafter referred to as \textit{'Client`}, to be conducted by \textit{'Pr0b3`}, hereinafter referred to as \textit{'Tester/Testers`}.
\textbf{Rules of Engagement} refer to broad agreed upon tactics of how the \textit{tester} will engage in analysis and \textbf{Scoping} refers to how much of the technical infrastructure of the \textit{client} can be leveraged for the Vulnerability Assessment.

This document covers general Rules of Engagement in a section followed by another section with Scoping. The Scoping section is further divided into four following parts:
\begin{itemize}
	\item{Network Penetration Test}
	\item{Wireless Penetration Test}
	\item{Web Application Penetration Tests}
	\item{Penetration Tests Leveraging Social Engineering Attacks}
\end{itemize}
Each subsection of scoping reflects a broad area in penetration assessment and might/might not overlap.

\section{Rules of Engagement}
\begin{itemize}
	\item{The agreed upon point of contact from the \textit{client} is \textbf{Mr.\ Matt Mason, Chief Engineer} and the preferred form of communication is through UF Slack Server. The point of contact will also serve as Emergency Point of Contact in case of any situation involving law enforcement authorities arising directly from assessment with faithful adherence to this document. Also, Mr. Matt Mason is fully authorized to hire \textit{testers} to conduct such a vulnerability assessment/penetration test that shall result from this agreement.}
	\item{\textit{Client} has no affiliation with any third party provider within the assessment scope that \textit{testers} need to know of and all the systems being tested are owned by \textit{Client} who has full authorization to perform tests on them. Mr. Matt Mason hereby authorizes the \textit{testers} to perform vulnerability assessment/penetration tests with faithful adherence to rules and scope mentioned within this document.}
	\item{The \textit{client} can ask \textit{testers} to stop the tests without any further explanation or notice.}
	\item{The vulnerability assessment shall be conducted from \textit{Sept 8, 2020} till \textit{Dec. 3, 2020} with no particular timing restrictions except 30 minutes prior and after as well as during published schedule of UFSIT meetings.}
	\item{Agreed upon method of assessment is \textit{Remote via NetLab} where Virtual Machines shall be provided for by the \textit{client}.}
	\item{The \textit{testers} are required to handle all data from the assessment with strongest form of encryption and confidentiality and agree to not employ any method that might break encryption/hashing of confidential data exfiltrated from the \textit{client}. All the data must be transferred using Secure File Transfer Protocol {SFTP} and copy of all encryption keys are required to be presented to the \textit{client} prior to any form of assessment.}
	\item{The \textit{testers} agree to provide regular status updates to the point of contact via status meetings.}
	\item{The \textit{client} provides no explicit guarantee that there will not be any mitigation attempts in case of detected infiltration during the assessment or otherwise.}
	\item{In the event of any of the systems owned by \textit{client} being in jeopardy of operational damage, the \textit{testers} are required to inform the \textit{client} ASAP.}
	\item{The Penetration Assessment resulting from this agreement is not intended to fulfill any compliance requirement for the \textit{client} imposed by any private, public or government entity.}
	\item{The \textit{client} provides no permission for physical penetration assessment and thus rogue devices within the \textit{client} facility are out of scope.}
	\item{None of the systems in scope are physically present outside Florida State and hence adhere only to laws under Florida State Legislature.}
	\item{In case of a remote infiltration, \textit{client} authorizes \textit{testers} to assume permission to gain highest level of administrative privilege and move laterally as well as horizontally within the network and under strict adherence of scope.}
\end{itemize}


\section{Scoping}
\subsection{Network Penetration Tests}
\begin{itemize}
	\item{The network in scope of assessment is 172.30.0.0/24 with f4rmc0rp.com hosted at 172.30.0.128.}
	\item{172.30.0.1 is the IP allocated to the router with external access, this is \textbf{out of scope}.}
	\item{There is no VPN which the \textit{client} hosts and thus VPNs are out of scope.}
	\item{The ISP providing public internet access to \textit{client} does not need to be consulted as much of assessment shall be aimed within \textit{client} network.}
	\item{In the event of successful penetration into the network, the Email infrastructure even through in scope, shall not be put under operational jeopardy.}
\end{itemize}

\subsection{Wireless Penetration Test}
\begin{itemize}
	\item{Any peripheral devices like routers, printers, cameras (excluding web cameras mounted on an active system which is within scope) etc.\ are \textbf{out of scope} for the wireless assessment.}
	\item{No guest network access shall be provided to the \textit{testers}.}
\end{itemize}

\subsection{Web Application Penetration Tests}
\begin{itemize}
	\item{A small web app hosted at f4rmc0rp.com is in scope.}
	\item{Source code for the web applications will not be provided.}
	\item{Fuzzing on the web applications are in scope.}
	\item{Role based testing as well as credentialed scans are \textbf{out of scope.}}
	\item{Any documentation/source code shall not be provided explicitly for the purposes of web applications testing but in the event that such an asset is found in network during the assessment, it can be leveraged to fullest extent. Such an asset is the confidential property of \textit{client} and shall be treated like any other private information found within the network.}
\end{itemize}

\subsection{Penetration Tests Leveraging Social Engineering Attacks}
Any sort of assessment involving social engineering tactics are \textbf{out of scope} except phishing attempts against Mr.\ Phineas up to reasonable extent.

\vspace*{\fill}
\signature{Mr.\ Matt Mason}{Chief Engineer, F4rmc0rp}\hfill\signature{Mr.\ Naman Arora}{Penetration Tester, Pr0b3 LLC}

\newpage

\begin{tcolorbox}[enhanced,attach boxed title to top left={yshift=-3mm,yshifttext=-1mm}, before upper={\parindent4em},
		colback=aliceblue,colframe=blue,colbacktitle=blue,
		title=Rationale,fonttitle=\large\bfseries,
	boxed title style={size=small,colframe=blue} ]
	The inclusion of most of the points in Rules of Engagement can be justified as bare minimum in the context of providing authorization. For example, all the points related to Mr. Matt Mason explicitly state who the point of contact is and whether he has the authority to grant permissions for the assessment. Some points answer the basic questions about when, where and how. One of the key aspects of engagement is the handling of the client data (if/when exfiltrated), which should be handled confidentially and securely. Other rules deals with regular status meetings, operational damage to systems (if any) and the extent up to which the testing can proceed in case of successful penetration. Agreement also clearly specifies it cant be used for satisfying compliance requirements and identifies the legal jurisdiction. A more fine grained approach towards rules might prove to take away from a real world attack simulation. \par
Scoping is subdivided into four parts, each identifying the extent to which the systems may be leveraged. These are, in a way, re-statement of the facts put forth by Mr. Mason during the meetings.
\end{tcolorbox}

\end{document}
