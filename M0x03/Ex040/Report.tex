\documentclass[10pt,a4paper]{article}

\usepackage[a-1b]{pdfx} %high quality pdf
\usepackage{datetime}
\usepackage{numprint}
\usepackage{palatino}
\usepackage{authblk}
\usepackage[margin=0.75in]{geometry}
\usepackage{listings}
\usepackage{hyperref}
\usepackage{graphicx}
\usepackage{multicol}

\pdfgentounicode=1

%paragraph indentation
\setlength{\parindent}{2em}
%\setlength{\parskip}{1em}
\renewcommand{\baselinestretch}{1}

\begin{document}

\nplpadding{2}
\newdateformat{isodate}{\THEYEAR-\numprint{\THEMONTH}-\numprint{\THEDAY}}

\title{Penetration Testing Report for Ex040\\ \large{\textit{Codename: `Wireshark`}}}
\author{Naman Arora}
\date{\isodate\today}

\maketitle
\section{Technical Report}
\subsection{Findings}
Both the domains \textit{plunder.pr0b3.com} and \textit{ns.f4rmc0rp.com} are reachable within a max of two hops distance.\\\\
The KEY006 can be seen in echo requests from the gateway router at \textit{`172.24.0.1`} to the kali machine.

\subsection{Attack Narrative}

\subsubsection{Task 01}
The very first task was to \textit{`traceroute`} the \textit{plunder.pr0b3.com} server.
\begin{lstlisting}[language=bash]
	>> sudo traceroute -I plunder.pr0b3.com
\end{lstlisting}
\begin{figure}[h!]
	\includegraphics[width=\textwidth]{pics/Screenshot from 2020-09-30 03-25-00.png}
	\caption{Traceroute to \textit{plunder.pr0b3.com}}
\end{figure}
The \textit{`traceroute`} functionality sends out 3 packets for each new TTL field with TTL ranging from 1-8.
Thus, in this case, we see 24 ping request packets.
The maximum ping requests that it requires for tracing the host was 3, as shown in the figure.
The \textit{`-m`} option is used to specify max TTL which happens to be 3, in case of \textit{plunder.pr0b3.com}.

\newpage
\subsubsection{Task 02}
The second task was to trace the route for the \textit{`ns.f4rmc0rp.com`}, which was done by,
\begin{lstlisting}[language=bash]
	>> sudo traceroute -I ns.f4rmc0rp.com
\end{lstlisting}
\begin{figure}[h!]
	\includegraphics[width=\textwidth]{pics/Screenshot from 2020-09-30 03-25-29.png}
	\caption{Traceroute to \textit{ns.f4rmc0rp.com}}
\end{figure}
Again, like the above case, the maximum ping requests that would have traced the route till \textit{`ns.f4rmc0rp.com`} was 2.
But the \textit{`traceroute`} sends 5 ping requests, that is, until it receives the response from the actual target.

\newpage
\subsubsection{Task 03}
The next task is to identify the course of action if the host or any intermediary node drops ICMP packets.
Since ICMP is a port-less protocol, a host cannot deny it just by blocking a port.
There needs to be a firewall set up (like IPTables for GNU/Linux) to do so.
In case such a situation exists, the \textit{man page} of \textit{`traceroute`} goes in depth about explaining different tactics that one can employ.
Alternate protocols include,
\begin{itemize}
	\item{TCP}\\
		\textit{`-T`} as the switch and \textit{`-p`} as port number.\\
		A well known method is TCP half open technique, similar to the \textit{`nmap's`} half open TCP scan.
	\item{UDP}\\
		Default method or \textit{`-U`} and \textit{`-p`} for port number.\\
		Unlikely destination ports are advised for higher chance of success.
	\item{UDP Light}\\
		Again with a default port of 53.
	\item{DCCP}\\
		Similar to TCP half open scan at unlikely ports using the DCCP (Datagram Congestion Control Protocol).
	\item{Raw packets}\\
		Raw crafted packets which might be suitable for the given situation and network needs.
\end{itemize}

\end{document}
