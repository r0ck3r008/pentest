\documentclass[10pt,a4paper]{article}

\usepackage[a-1b]{pdfx} %high quality pdf
\usepackage{datetime}
\usepackage{numprint}
\usepackage{palatino}
\usepackage{authblk}
\usepackage[margin=0.75in]{geometry}
\usepackage{hyperref}
\usepackage{graphicx}
\usepackage{titlesec}
\usepackage{listings}

\pdfgentounicode=1

\setlength{\parindent}{2em}
%\setlength{\parskip}{1em}
\renewcommand{\baselinestretch}{1.0}

\begin{document}

\nplpadding{2}

\title{Penetration Testing Report for Ex140\\ {\large \textit{Codename: `MobileAppTest'}}}
\author{Naman Arora}
\date{\today}

\maketitle
\section*{Technical Report}

\subsection*{Finding: MS17-010 (Eternal Blue)}
        \begin{itemize}
                \item{Risk Rating: 10.0}\\
                        The host \textit{bdc.f4rmc0rp.com} is vulnerable to \href{https://docs.microsoft.com/en-us/security-updates/SecurityBulletins/2017/ms17-010}{MS17-010} vulnerability.
                \item{Vulnerability Description}\\
        \textit{EternalBlue} is a cyberattack exploit that was developed by NSA (National Security Agency) in 2017.
                It allows the malicious actors to allow remote execution of arbitrary commands and gain access to the remote host by sending uniquely crafted messages.
                This tool specifically exploits the Server Message Block version which is also known as SMBv1 protocol that is present in Microsoft's Windows Operating System.
                SMBv1 is a file sharing protocol that allows remote access of files on a server. 

        Due to its nature, if one machine that is connected to the internet is infected via EnternalBlue, the whole network could be compromised.
        This makes the network harder to recover as all the machines in the network had to be brought down to remediate. The Malware was patched in MS17-010 
                \item{Confirmation Method}\\
                        The host is a \textit{Windows Server 2016 x64 System} and can be confired as vulnerable to \textit{MS17-010} by visiting the MS official advisory.
                        Another confirmation can be \textit{Metasploit Auxillary Scanner} (Fig. \ref{metasploit}).
                \item{Remediation}\\
                        The system can be patched to latest version provided by Microsoft.
        \end{itemize}

\section*{Attack Narrative}
The host \textit{bdc.f4rmc0rp.com's} availability can be scanned using \textit{nmap} on \textit{devbox.f4rmc0rp.com} using the \textit{SSH} access obtained earlier.
\textit{Metasploit} can then be used to scan and then exploit the server to create a user \textit{John} annd add it to the \textit{``Domain Admins''} group (Fig \ref{useradd} and \ref{da_add}).
Once the user is added, the user can then be used to login to the \textit{pdc.f4rmc0rp.com} to exfiltrate corporate secrets from the \textit{DC} (Fig. \ref{smblist}, \ref{sysvol} and \ref{text}).

\begin{figure}[!htbp]% [!hb] forces image to be placed at that position
    \centering
    \includegraphics[width=\columnwidth]{pics/likely_vuln.png}
    \caption{Metasploit auxiliary \textit{MS17-010} Scanner }
    \label{metasploit}
\end{figure}

\begin{figure}[!htbp]% [!hb] forces image to be placed at that position
    \centering
    \includegraphics[width=\columnwidth]{pics/john_useradd.png}
    \caption{John User Add}
    \label{useradd}
\end{figure}

\begin{figure}[!htbp]% [!hb] forces image to be placed at that position
    \centering
    \includegraphics[width=\columnwidth]{pics/john_admins_add.png}
    \caption{Add John to \textit{`Domain Admins'}}
    \label{da_add}
\end{figure}

\begin{figure}[!htbp]% [!hb] forces image to be placed at that position
    \centering
    \includegraphics[width=\columnwidth]{pics/john_smblist.png}
    \caption{List of Shares in \textit{bdc.f4rmc0rp.com}}
    \label{smblist}
\end{figure}


\begin{figure}[!htbp]% [!hb] forces image to be placed at that position
    \centering
    \includegraphics[width=\columnwidth]{pics/john_sysvol_access.png}
    \caption{Access to \textit{SYSVOL} in \textit{bdc.f4rmc0rp.com}}
    \label{sysvol}
\end{figure}


\begin{figure}[!htbp]% [!hb] forces image to be placed at that position
    \centering
    \includegraphics[width=\columnwidth]{pics/sacred_text.png}
    \caption{Sacred Text}
    \label{text}
\end{figure}

\newpage
\bibliographystyle{plain}
\bibliography{references} % see references.bib for bibliography management
\end{document}
