\documentclass[10pt,a4paper]{article}

\usepackage[a-1b]{pdfx} %high quality pdf
\usepackage{datetime}
\usepackage{numprint}
\usepackage{palatino}
\usepackage{authblk}
\usepackage[margin=0.75in]{geometry}
\usepackage{listings}
\usepackage{hyperref}
\usepackage{graphicx}
\usepackage{multicol}

\pdfgentounicode=1

%paragraph indentation
\setlength{\parindent}{2em}
%\setlength{\parskip}{1em}
\renewcommand{\baselinestretch}{1}

\begin{document}

\nplpadding{2}
\newdateformat{isodate}{\THEYEAR-\numprint{\THEMONTH}-\numprint{\THEDAY}}

\title{Penetration Testing Report for Ex050\\ \large{\textit{Codename: `Nmap`}}}
\author{Naman Arora}
\date{\isodate\today}

\maketitle
\section{Technical Report}
\subsection{Findings}
The following services were found to be active on the \textit{www.f4rmc0rp.com} host,
\begin{center}
\begin{tabular}{ c c c c }
	Port number & State & Service & Version\\\hline
	22/tcp & open & ssh & OpenSSH 7.9p1 \\
	53/tcp & open & domain & ISC BIND 9.11.5\\
	80/tcp & open & http & Apache httpd 2.4.38\\
	443/tcp & open & ssl/ssl & Apache httpd \\
	2121/tcp & open & ftp & vsftpd 2.3.4\\
	40/udp & open$|$filtered & unknown & unknown\\
\end{tabular}
\end{center}
The following vulnerabilities were found during the scanning phase,
\begin{itemize}
	\item{Vsftpd 2.3.4}
		\begin{enumerate}
			\item[$\ast$]{\href{https://www.exploit-db.com/exploits/17491}{Backdoor Command Execution \color{red}(Exploit!)}}
			\item[$\ast$]{Anonymous Login Possible (Mis-Configuration)}
		\end{enumerate}
	\item{OpenSSH 7.9}
		\begin{enumerate}
			\item[$\ast$]{\href{https://www.cvedetails.com/cve/CVE-2019-6111/}{Directory/Subdirectory traversal of the SCP client with arbitrary overwrite capability.}}
			\item[$\ast$]{\href{https://www.cvedetails.com/cve/CVE-2019-6110/}{Malicious server can manipulate client output and transfer additional files.}}
			\item[$\ast$]{\href{https://www.cvedetails.com/cve/CVE-2019-6109/}{Similar in nature to CVE-2019-6110.}}
			\item[$\ast$]{\href{https://www.cvedetails.com/cve/CVE-2018-20685/}{Bypass restriction and modify file permissions.}}
		\end{enumerate}
	\item{Apache 2.4.38}
		\begin{enumerate}
			\item[$\ast$]{\href{https://www.cvedetails.com/cve/CVE-2019-10098/}{Redirection to unexpected URL within request URL.}}
			\item[$\ast$]{\href{https://www.cvedetails.com/cve/CVE-2019-10097/}{Stack buffer overflow or NULL pointer dereference.}}
			\item[$\ast$]{\href{https://www.cvedetails.com/cve/CVE-2019-10092/}{XSS}}
			\item[$\ast$]{\href{https://www.cvedetails.com/cve/CVE-2019-10082/}{Read after free.}}
			\item[$\ast$]{\href{https://www.cvedetails.com/cve/CVE-2019-10081/}{Memory overwrite leading to crash.}}
			\item[$\ast$]{\href{https://www.cvedetails.com/cve/CVE-2019-0220/}{Multiple consecutive '/' lead to implicit mismatch in parts of server processing thus, crashes.}}
			\item[$\ast$]{\href{https://www.cvedetails.com/cve/CVE-2019-0217/}{Bypass configured access control restriction using valid credentials.}}
			\item[$\ast$]{\href{https://www.cvedetails.com/cve/CVE-2019-0215/}{Bypass configured access control restrictions.}}
			\item[$\ast$]{\href{https://www.cvedetails.com/cve/CVE-2019-0211/}{Code execution in less privileged child leads to root execution in parent, only UNIX systems.}}
			\item[$\ast$]{\href{https://www.cvedetails.com/cve/CVE-2019-0197/}{Upgrade request from http/1.1 to htpps/2 may lead to misconfiguration or crash.}}
			\item[$\ast$]{\href{https://www.cvedetails.com/cve/CVE-2019-0196/}{Use after free leading to incorrect processing of a request.}}
		\end{enumerate}
	\item{Isc Bind 9.11.3}
		\begin{enumerate}
			\item[$\ast$]{\href{https://www.cvedetails.com/cve/CVE-2018-5741/}{Misleading configuration documentation.}}
			\item[$\ast$]{\href{https://www.cvedetails.com/cve/CVE-2018-5740/}{Internal assertion failure by using `deny-answer-aliases` feature.}}
			\item[$\ast$]{\href{https://www.cvedetails.com/cve/CVE-2017-3145/}{Use after free resulting in assertion failure or crash.}}
			\item[$\ast$]{\href{https://www.cvedetails.com/cve/CVE-2017-3143/}{Accept unauthorized dynamic update.}}
			\item[$\ast$]{\href{https://www.cvedetails.com/cve/CVE-2017-3142/}{Authentication circumvention leading to bogus AXFR request.}}
			\item[$\ast$]{\href{https://www.cvedetails.com/cve/CVE-2017-3141/}{Privilege escalation due to unquoted path in Windows.}}
			\item[$\ast$]{\href{https://www.cvedetails.com/cve/CVE-2017-3140/}{Endless loop if configured to use Response Processing Zones in some cases.}}
			\item[$\ast$]{\href{https://www.cvedetails.com/cve/CVE-2016-9444/}{Denial of Service via crafted DS resource record in answer.}}
			\item[$\ast$]{\href{https://www.cvedetails.com/cve/CVE-2016-9131/}{Denial of Service via malformed response to RTYPE ANY query.}}
			\item[$\ast$]{\href{https://www.cvedetails.com/cve/CVE-2016-8864/}{Denial of Service via DNAME record in answer section of recursive query.}}
			\item[$\ast$]{\href{https://www.cvedetails.com/cve/CVE-2016-6170/}{Denial of service due to large AXFR and UPDATE message.}}
			\item[$\ast$]{\href{https://www.rapid7.com/db/modules/auxiliary/dos/dns/bind_tsig}{Denial of Service via crafted query \color{red}(Exploit!).}}
			\item[$\ast$]{\href{https://www.cvedetails.com/cve/CVE-2016-2775/}{Denial of Service via long request when \textit{`iwres`} is enabled.}}
		\end{enumerate}
	\item{\color{red} TODO! Check vulns for port 40!}
\end{itemize}

\subsection{Attack Narrative}

\subsubsection{Task 01}
The first task was to run TCP Version scan on the host \textit{www.f4rmc0rp.com}.\\
This was done using,\\
\begin{lstlisting}[language=bash]
	$> nmap -sV www.f4rmc0rp.com
\end{lstlisting}
This returned 5 running services on the host and took approximately 10 seconds.
\begin{figure}[h!]
	\includegraphics[width=\textwidth]{pics/Screenshot from 2020-09-29 14-44-28.png}
	\caption{TCP Version Scan}
\end{figure}
On running it with the aggressive switch \textit{-A}, it returns details about the debian based host.\\
Most notably, port 2121 is misconfigured and can be accessed anonymously.
\begin{figure}[h!]
	\includegraphics[width=\textwidth]{pics/Screenshot from 2020-09-29 14-40-09.png}
	\caption{TCP Version Scan Aggressive Mode}
\end{figure}
On logging in the ftp service anonymously, \textit{`hereisafile`} can be successfully exfiltrated which has no data.

\newpage
\subsubsection{Task 02}
The second task was to run a UDP version scan on the host \textit{`www.f4rmc0rp.com`} from ports 1-256.\\
This was done using the following command,\\
\begin{lstlisting}[language=bash]
	$> sudo nmap -sU www.f4rmc0rp.com -p 1-256
\end{lstlisting}
This scan takes significantly longer than the TCP counterpart, about 4 minutes.\\
Two ports, 1 previously unknown, were discovered.\\
The port 40 is notably suspicious as this is not a well known port thus on running scan on this and noting the traffic in the wireshark, KEY007 is revealed.
\begin{figure}[h!]
	\includegraphics[width=\textwidth]{pics/Screenshot from 2020-09-29 13-40-44.png}
	\caption{UDP Version Scan}
\end{figure}

\newpage
\subsection{Risk Analysis}
Some of the vulnerabilities presented in the \textit{`findings`} sections can be leveraged to reliably infiltrate the host.\\
A couple of vulnerabilities have pre-crafted exploits readily available with very low bar of complexity.\\
Team meeting URL is \href{https://ufl.zoom.us/j/97385390472}{https://ufl.zoom.us/j/97385390472}.

\end{document}
