\documentclass[10pt,a4paper]{article}

\usepackage[a-1b]{pdfx} %high quality pdf
\usepackage{datetime}
\usepackage{numprint}
\usepackage{palatino}
\usepackage{authblk}
\usepackage[margin=0.75in]{geometry}
\usepackage{hyperref}
\usepackage{graphicx}
\usepackage{titlesec}
\usepackage{listings}

\pdfgentounicode=1

\setlength{\parindent}{2em}
%\setlength{\parskip}{1em}
\renewcommand{\baselinestretch}{1.0}

\begin{document}

\nplpadding{2}

\title{Penetration Testing Report for Ex130\\ {\large \textit{Codename: `EAPWireless'}}}
\author{Naman Arora}
\date{\today}

\maketitle
\section*{Technical Report}
\subsection*{Finding: Missing \textit{`ca-cert'} in TTLS Wireless Configuration on \textit{`f4rmc0rp-ddwrt-1'}}
\begin{itemize}
	\item{Risk Rating: 7.0}\\
		A malicious device may be successful is impersonating as \textit{`f4rmc0rp-ddwrt-1'}.
	\item{Vulnerability Description}\\
		A malicious device may pose as \textit{`f4rmc0rp-ddwrt-1'} and capture the \textit{Challenge} and \textit{Response} of a device trying to associate.
		The captured \textit{Challenge} and \textit{Response} pair may then be used to crack the user credentials for the access to internal wireless network.
	\item{Confirmation Method}\\
		A created rogue device broadcasting the same \textit{ESSID} on same channel as \textit{`f4rmc0rp-ddwrt-1'} without any certificates captures connection attempts.
	\item{Remediation}\\
		Use a trusted \textit{CA Certificate} in \textit{wpa\_configuration} to avoid devices impersonating \textit{`f4rmc0rp-ddwrt-1'}, refer \href{https://w1.fi/cgit/hostap/plain/wpa\_supplicant/wpa\_supplicant.conf}{Sample WPA configuration}.
\end{itemize}

\section*{Attack Narrative}
To attack the misconfiguration with the \textit{`f4rmc0rp-ddwrt-1'} router, one can use \textit{`airodump-ng'} to check which channel the transmission is on.
This can be done like,
\begin{lstlisting}[language=bash]
        >> airmon-ng check kill # To kill services that may interfere
	>> airodump-ng wlan0 # Wlan0 refers to the primary wireless NIC
\end{lstlisting}

Then, \textit{hostapd-wpe} can be used to impersonate the wireless router by defining the broadcast name in \textit{`/etc/hostapd-wpe/hostapd-wpe.conf'} and running,
\begin{lstlisting}[language=bash]
        >> hostapd-wpe /etc/hostapd-wpe/hostapd-wpe.conf
\end{lstlisting}
After some time, a message like Fig \ref{hostapd} will appear.
This shows a client has tried to associate with the fake endpoint and the \textit{Challenge} and \textit{Response} values have been now captured.
\begin{figure}[!htbp]% [!hb] forces image to be placed at that position
	\centering
	\includegraphics[width=\columnwidth]{pics/hostapd.png}
	\caption{\textit{hostapd-wpe} Capturing the Challenge and Response values}
	\label{hostapd}
\end{figure}

The passphrase for association with the real endpoint can then be cracked using \textit{`asleap'} using \textit{`rockyou'} wordlist, in this case \ref{asleap}.
\begin{figure}[!htbp]% [!hb] forces image to be placed at that position
	\centering
	\includegraphics[width=\columnwidth]{pics/brian_passwd.png}
	\caption{\textit{asleap} with \textit{`rockyou'}}
	\label{asleap}
\end{figure}
Once password is cracked, network services can be restarted as \ref{nm_restart} and appropriate configuration can be used to associate with the endpoint.
\begin{figure}[!htbp]% [!hb] forces image to be placed at that position
	\centering
	\includegraphics[width=\columnwidth]{pics/nm_restart.png}
	\caption{Network Manager and related service restart}
	\label{asleap}
\end{figure}

Access to internal website is then trivial (Fig \ref{root}, \ref{corp} and \ref{message}).
\begin{figure}[!htbp]% [!hb] forces image to be placed at that position
	\centering
	\includegraphics[width=\columnwidth]{pics/root.png}
	\caption{Access to Internally hosted webpage \textit{1/3}}
	\label{root}
\end{figure}
\begin{figure}[!htbp]% [!hb] forces image to be placed at that position
	\centering
	\includegraphics[width=\columnwidth]{pics/corp.png}
	\caption{Access to Internally hosted webpage \textit{2/3}}
	\label{corp}
\end{figure}
\begin{figure}[!htbp]% [!hb] forces image to be placed at that position
	\centering
	\includegraphics[width=\columnwidth]{pics/message.png}
	\caption{Access to Internally hosted webpage \textit{3/3}}
	\label{message}
\end{figure}

\end{document}
