\documentclass[10pt,a4paper]{article}

\usepackage[a-1b]{pdfx} %high quality pdf
\usepackage{datetime}
\usepackage{numprint}
\usepackage{palatino}
\usepackage{authblk}
\usepackage[margin=0.75in]{geometry}
\usepackage{hyperref}
\usepackage{graphicx}
\usepackage{titlesec}
\usepackage{listings}

\pdfgentounicode=1

\setlength{\parindent}{2em}
%\setlength{\parskip}{1em}
\renewcommand{\baselinestretch}{0.5}

\begin{document}

\nplpadding{2}

\title{Penetration Testing Report for Ex140\\ {\large \textit{Codename: `MobileAppTest'}}}
\author{Naman Arora}
\date{\today}

\maketitle
\section*{Technical Report}
\subsection*{Finding: Debug Symbols in Production Application}
\begin{itemize}
        \item{Risk Rating: 3.0}\\
        The \textit{.apk} released for the customer has been built in debug mode.
        \item{Vulnerability Description}\\
        A malicious actor might reverse engineer the code to find hidden vulnerabilities.
        A release mode removes the debug symbols which helps in obfuscating some vulnerabilities until they are fixed.
        \item{Confirmation Method}\\
        
        \item{Mitigation Strategy}
    \begin{figure}[h!]
    	\includegraphics[width=\textwidth]{pics/htpasswd.png}
            \caption{\textit{`htpasswd'} Access}
            \label{htpasswd}
    \end{figure}
\end{itemize}
\subsection*{Finding: Hardcoded Database Server Credentials}
\begin{itemize}
        \item{Risk Rating: }
        \item{Vulnerability Description}
        \item{Confirmation Method}
        \item{Mitigation Strategy}
\end{itemize}
\subsection*{Finding: }
\begin{itemize}
        \item{Risk Rating: }
        \item{Vulnerability Description}
        \item{Confirmation Method}
        \item{Mitigation Strategy}
\end{itemize}

\section*{Attack Narrative}
\end{document}
