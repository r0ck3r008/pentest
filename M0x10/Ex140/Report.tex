\documentclass[10pt,a4paper]{article}

\usepackage[a-1b]{pdfx} %high quality pdf
\usepackage{datetime}
\usepackage{numprint}
\usepackage{palatino}
\usepackage{authblk}
\usepackage[margin=0.75in]{geometry}
\usepackage{hyperref}
\usepackage{graphicx}
\usepackage{titlesec}
\usepackage{listings}

\pdfgentounicode=1

\setlength{\parindent}{2em}
%\setlength{\parskip}{1em}
\renewcommand{\baselinestretch}{1.0}

\begin{document}

\nplpadding{2}

\title{Penetration Testing Report for Ex140\\ {\large \textit{Codename: `MobileAppTest'}}}
\author{Naman Arora}
\date{\today}

\maketitle
\section*{Technical Report}
\subsection*{Finding: Debug Symbols in Production Application}
\begin{itemize}
        \item{Risk Rating: 3.0}\\
                The \textit{.apk} released for the customer has been built in debug mode.
        \item{Vulnerability Description}\\
                A malicious actor might reverse engineer the code to find hidden vulnerabilities.
                A release mode removes the debug symbols which helps in obfuscating some vulnerabilities until they are fixed.
        \item{Confirmation Method}\\
                The application can be reversed to reveal the \textit{`BuildConfig.java'} file with \textit{Debug mode} (Fig \ref{dbg}).
        \item{Mitigation Strategy}
                The release application should always be build in an explicit release mode \href{https://github.com/OWASP/owasp-masvs/blob/master/Document/0x12-V7-Code\_quality\_and\_build\_setting\_requirements.md}{MASVS}.
    \begin{figure}[h!]
    	\includegraphics[width=\textwidth]{pics/dbg.png}
            \caption{Debug Mode Enabled}
            \label{dbg}
    \end{figure}
\end{itemize}
\subsection*{Finding: Hardcoded Database Server Credentials}
\begin{itemize}
        \item{Risk Rating: 10.0}\\
                The database server credentials were hard-coded in the application source.
        \item{Vulnerability Description}\\
                A hard-coded string of credentials to the database server along with the URL of the database resource will lead a malicious actor to gain unauthorized access to that resource.
        \item{Confirmation Method}\\
                The reversing of the \textit{.apk} reveals a file inside the \textit{sources} named \textit{`ItemListActivity.java'} which contains the credentials (Fig \ref{creds}).
        \item{Mitigation Strategy}\\
                The application must never have hard-coded, instead, access tokens and authentications should be used by the application to gain access to remote resources \href{https://github.com/OWASP/owasp-masvs/blob/master/Document/0x09-V4-Authentication_and_Session_Management_Requirements.md}{MASVS}.
    \begin{figure}[h!]
    	\includegraphics[width=\textwidth]{pics/creds.png}
        \caption{Credentials in \textit{`ItemListActivity.java'}}
        \label{creds}
    \end{figure}
\end{itemize}

\section*{Attack Narrative}
The application once obtained from \textit{`http://www.f4rmc0rp.com/apps/F4rmC0rp.apk'}, can be reversed easily using the tool \textit{jadx}.
The following command can be used for reversing,
\begin{lstlisting}[language=bash]
        >> jadx -d ${PWD}/out ${PWD}/F4rmC0rp.apk
\end{lstlisting}
The directory \textit{`out`} will then contain reversed sources to the application.

The meeting URL \href{https://ufl.zoom.us/j/6521580530}{https://ufl.zoom.us/j/6521580530}.
\end{document}
