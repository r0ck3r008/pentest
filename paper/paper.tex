\documentclass[conference]{IEEEtran}
\IEEEoverridecommandlockouts
% The preceding line is only needed to identify funding in the first footnote. If that is unneeded, please comment it out.
%\usepackage{cite}
\usepackage{amsmath,amssymb,amsfonts}
\usepackage{algorithmic}
\usepackage{graphicx}
\usepackage{textcomp}
\usepackage{xcolor}
\usepackage[utf8]{inputenc}
\usepackage[english]{babel}
\usepackage[
backend=biber,
style=numeric-comp,
]{biblatex}
\addbibresource{refs.bib}

\begin{document}

\title{KademGo: Kademlia DHT Implementaion}

\author{\IEEEauthorblockN{Naman Arora}
        \IEEEauthorblockA{Graduate Student, CISE Department}
        University of Florida\\
        naman.arora@ufl.edu }

\maketitle

\begin{abstract}
        The document describes an attempt to implement Kademlia, a Peer to Peer Distributed Hash Table algorithm in the form of a GoLang library.
        The project, \textit{KademGo}, aims to provide a flexible, efficient and easy to use overlay network complete with discovery, node maintenance and object transportation.
\end{abstract}

\section*{Introduction}
Maymounkov \textit{et. al} \cite{kademlia} in 2002 published an algorithm for facilitating communication between decentralized peers in an asynchronous fashion.
The algorithm is a type of \textit{Distributed Hash Table} \cite{wiki-dht} technique based on XOR distance metric.
The function of XOR distance metric is to define a parameter of logical distance between two nodes which is efficient in calculation.
It does not represent any physical distance.

The project \textit{KademGo} \cite{kademgo} is a vanilla implementation of this algorithm.
It aims to provide a small (just about 800 lines of code), efficient and easy to use implementation ready to be imported and accessed as an API in any project of choice.

\section*{The Original Task}
The project, originally, was aimed to be a protocol implementation of Overbot presented by Starnberger \textit{et. al} \cite{overbot} in 2008.
The Overbot protocol is an ambitious project to implement a secure and reliable way of communication between a \textit{BotMaster} and the \textit{BotNet}.
It is based upon the Kademlia algorithm for managing the multitude of peers.
Being a very ambitious project and in the interest of time, the project was turned into just the implementation of Kademlia in a form of an API.
The Overbot part of the project is postponed for future implementations which will now be built up on \textit{KademGo}.

\section*{The Work Input}
The most important work done to create \textit{KademGo} is understanding how actual P2P systems work.
P2P systems are specially tricky to create since all of the peers actually drive each other.
A message from \textit{Peer A} may induce a transition in state of \textit{Peer B} but only in a sand-boxed enviornment that \textit{Peer B} creates specifically for \textit{Perr A}.
This means a lot of emphasis goes in deciding whether a neighbor, currently at the top of the list of most frequent contacts, is alive or not.
In case of a dead neighbor, a swift detection is necessary for contingency actions to be taken.
This means a lot of code, currently about 2 modules worth, in this case \textit{(lru.go and nbrmap.go)}, goes in this very process.

The next important pieces of contribution is specific to Kademlia.
Kademlia requires each node lookup to be a \textit{RPC} from the node querying to the nearest neighbor with a certainty that either the address of the node being looked up is returned or at least \textit{`k'} of its nearest neighbors are returned.
This requires careful consideration for fast lookup buckets where the neighbors are stored.
Looking back, KademGo has been a great lesson in practical \textit{Algorithms} and \textit{Data Structures}.

The third and one of the most overlooked inputs when it comes to describe a project is mastering the programming language itself.
GoLang turned out to be huge asset. Having absolutely no experience in it at all, was a grind to understand.
The concurrency model of this language is at par with the likes of \textit{Earlang} which now has been modeled into \textit{Elixir}.
\textit{Go Channels} are fascinating and with the ability to trivially spin up thousands of \textit{Go Routines}, it is a delight working with this language.
It took a lot of time transitioning from a \textit{C} like memory management and OS threads model to a \textit{Green thread} model employed by Go.

\section*{The Produced Output}
The produced project is open for community contributions and now is aimed at really providing value to anyone who needs a quick P2P library based on a robust algorithm.
Documentation of the Go code is generally accessed using the \textit{Godoc} command \cite{godoc} with KademGo being no exception to that.
A quick example to access the documentation is explicitly stated in the \textit{README} of the \textit{KademGo's} repository.

\section*{Self Assessment}
A quick self analysis leads to the conclusion even after all the work that went into the project, a throughly benchmarked project was not created in the given time frame.
Eluding to the fact that the project is still ongoing and knows no semester boundaries, probably this is the only way to maintain a higher standard of code quality of the project.
The fact that this project is a networking library, on which some other project will be built, each and every component of the code has to be thought through to make it as future proof and modular as possible.
The agreed upon task turned out to be much more nuanced and seeking responsible coding than initially thought, especially for a single developer along with other graduate level courses.
Grading the project, as untested as it stands, can be based on the work and effort put in, the value the project may provide and the level of efficiency and future-proofing, which leads to a self assessed score of 90/100 points with a breakup of 40 for future value it may provide to another project, 30 for implementation nuances and rest 20 for effort.

\printbibliography

\end{document}
