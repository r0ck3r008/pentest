\documentclass[10pt,a4paper]{article}

\usepackage[a-1b]{pdfx} %high quality pdf
\usepackage{datetime}
\usepackage{numprint}
\usepackage{palatino}
\usepackage{authblk}
\usepackage[margin=1in]{geometry}
\usepackage[most]{tcolorbox} % Rational colorbox
\usepackage{listings}

\pdfgentounicode=1

%paragraph indentation
\setlength{\parindent}{4em}
%\setlength{\parskip}{1em}
\renewcommand{\baselinestretch}{1.25}

\begin{document}

\nplpadding{2}
\newdateformat{isodate}{\THEYEAR-\numprint{\THEMONTH}-\numprint{\THEDAY}}

\title{Penetration Testing Report for Ex010\\ \large{\textit{Codename: `Kali Netlab`}}}
\author{Naman Arora}
\date{\isodate\today}

\maketitle
\section{Attack Narrative}
\subsection{Task 01}
The very first task was to use the Unix shell command \textit{'find`} to find the key in the file system of the Kali attack box.
The following approach was employed,
\begin{itemize}
	\item{Find out how the command \textit{'find`} works:}\\
		This can be done using the \textit{manpages}.
		\begin{lstlisting}[language=bash]
		$> man find
		\end{lstlisting}
		Useful attribute is \textit{-name pattern} which is capable of doing a search using the name of the file.
	\item{Searching for the Key in the File system:}\\
		Since the template of Keys is known beforehand, we can use the following command,
		\begin{lstlisting}[language=bash]
		$> find / -name "*KEY*"
		\end{lstlisting}
		Which yields the, among other results, a file named \textbf{\textit{'KEY001:thZp0CuipB5dlHSIBIujUg==`}} under the \textit{'/boot/grub`} directory.
\end{itemize}

\subsection{Task 02}
As mentioned in the objective document for the exercise, the manpage for the command to be used includes a \textit{`only yourself`} clause.
To leverage this, the following steps were employed,
\begin{itemize}
	\item{Find out more about the \textit{`man`} command:}\\
		On executing,
		\begin{lstlisting}[language=bash]
		$> man man
		\end{lstlisting}
		There is an attribute \textit{`--global-appropose`} or \textit{`-K`} for short that can search in all the man pages for a perticular string.
	\item{Finding the right command:}\\
		On executing,
		\begin{lstlisting}[language=bash]
		$> man -K "only yourself"
		\end{lstlisting}
		the first result that comes up is the \textit{`ps`} command with the \textit{`a`} attribute.
		On reading it a bit further, the combination of \textit{`aux`} attributes look reasonable for execution with verbose enough output.

	\item{Finding the key:}\\
		On executing
		\begin{lstlisting}[language=bash]
		$> ps aux | grep -i key
		\end{lstlisting}
		the key, \textbf{\textit{`KEY002:Y/IWt0eS4/73sk3qaFn08g==`}} is revealed.
	\item{Finding the key process:}\\
		For investigating further, the process that runs as key002 can be found. On executing,
		\begin{lstlisting}[language=bash]
		$> lsof -p `pidof KEY002:Y/IWt0eS4/73sk3qaFn08g==`
		\end{lstlisting}
		This reveals a \textit{`.key-process`} under \textit{`/home/kali`} as the executing process.
\end{itemize}

\end{document}
